\documentclass[xcolor=dvipsnames]{beamer}

\setbeamertemplate{caption}[numbered]

\useinnertheme{circles}
\useoutertheme{infolines}
\usefonttheme{structurebold}

\colorlet{theme}{blue}
\colorlet{coltheme}{theme!40!black}

% Colortheme monochromatic
\setbeamercolor{palette primary}{bg=coltheme!65, fg=white}
\setbeamercolor{palette secondary}{bg=coltheme!75, fg=white}
\setbeamercolor{palette tertiary}{bg=coltheme!85, fg=white}
\setbeamercolor{palette quaternary}{bg=coltheme!50, fg=white}
\setbeamercolor{subsection in head/foot}{bg=coltheme, fg=white}

\setbeamercolor{structure}{fg=coltheme!90} % itemize, enumerate, title, etc
\setbeamercolor{section in toc}{fg=coltheme!80} % TOC section

% selbstdefinierte Überschrift
\newcommand\beamheading[1]{%
  \par\bigskip
  {\Large\color{coltheme!90}\bfseries#1}
  \par\smallskip}

% Einfügen einer Titelseite vor jeder neuen Section
\AtBeginSection[]{
	\begin{frame}
	\vfill
	\centering
	{\huge\usebeamercolor[fg]{title}\bfseries\insertsectionhead\par}
	\vfill
	\end{frame}
}

% Längen und Abstände festlegen 
\setlength{\parindent}{0pt} 
%\setlength{\parskip}{\medskipamount}

\setbeamersize{
text margin left=5mm,
text margin right=5mm
}

\usepackage[ngerman]{babel}
\usepackage[T1]{fontenc}
\usepackage{verbatim}
\usepackage{amsmath, amssymb, amsfonts}

% Schriftart
\usepackage{helvet}
%\usepackage{mathptmx}

% Verlinkung
\usepackage{hyperref}
\hypersetup{
	colorlinks = true,
	linkcolor=,
	urlcolor = coltheme,
	pdftitle = {LaTeX-Tutorium}
}

% Literaturverzeichnis
\usepackage[backend=biber]{biblatex}
\addbibresource{literatur.bib}

% Quellcode-Einbindung
\usepackage{listings}		% Darstellung von Quellcode mit den Umgebungen {lstlisting}, \lstinline und \lstinputlisting
\lstset{
literate=					% Ermöglicht Umlaute innerhalb von Listing-Umgebungen
	{Ö}{{\"O}}1
	{Ä}{{\"A}}1
	{Ü}{{\"U}}1
	{ß}{{\ss}}1
	{ü}{{\"u}}1
	{ä}{{\"a}}1
	{ö}{{\"o}}1
}
% Einbindung von Code Snippets
\lstset{
	aboveskip=-15pt,
	belowskip=10pt,
	backgroundcolor=\color{coltheme!5},   % choose the background color; you must add \usepackage{color} or \usepackage{xcolor}; should come as last argument
	basicstyle=\footnotesize\ttfamily,        % the size of the fonts that are used for the code
	breakatwhitespace=false,         % sets if automatic breaks should only happen at whitespace
	breaklines=true,                 % sets automatic line breaking
	captionpos=t,                    % sets the caption-position to (b) bottom or (t) top
	commentstyle=\color{cyan},    	% comment style
	deletekeywords={...},            % if you want to delete keywords from the given language
	escapeinside={\%*}{*)},          % if you want to add LaTeX within your code
	escapeinside={(*@}{@*)},
	extendedchars=true,              % lets you use non-ASCII characters; for 8-bits encodings only, does not work with UTF-8
	xleftmargin=\parindent,	
	frame=l,	                	% "single" adds a frame around the code; "none"
	keepspaces=true,                 % keeps spaces in text, useful for keeping indentation of code (possibly needs columns=flexible)
	keywordstyle=\color{theme!70},   % keyword style
	identifierstyle=\color{coltheme},	
	language=[LaTeX]TeX,             % the language of the code
	morekeywords={*,nomenclature},   % if you want to add more keywords to the set
	numbers=left,                    % where to put the line-numbers; possible values are (none, left, right)
	numbersep=5pt,                   % how far the line-numbers are from the code
	numberstyle=\tiny\color{gray}, % the style that is used for the line-numbers
	rulecolor=\color{coltheme},         % if not set, the frame-color may be changed on line-breaks within not-black text (e.g. comments (green here))
	showspaces=false,                % show spaces everywhere adding particular underscores; it overrides 'showstringspaces'
	showstringspaces=false,          % underline spaces within strings only
	showtabs=false,                  % show tabs within strings adding particular underscores
	stepnumber=1,                    % the step between two line-numbers. If it's 1, each line will be numbered
	stringstyle=\color{lime!60},     % string literal style
	tabsize=2,	                   % sets default tabsize to 2 spaces
	title=\lstname                   % show the filename of files included with \lstinputlisting; also try caption instead of title
}

% Titelseite ------------------------------------

\title[\LaTeX{}-Tutorium]{Yet another \LaTeX{}-Tutorial...}
\subtitle{...using \LaTeX{}}
\author[Tim P.]{Tim Prüß}
\institute[DHBW]{DHBW Ravensburg Campus Friedrichshafen}
\date{\today}

% -----------------------------------------------
% Das folgende Dokument ist nicht ganz leicht zu lesen,
% weil ich das erste Mal mit beamer-LaTeX arbeite
% Ich entschuldige mich schonmal dafür ^^

\begin{document}

%% Titelseite
\begin{frame}
	\titlepage
\end{frame}

%% Inhaltsverzeichnis
\begin{frame}
	\tableofcontents[hideallsubsections]
\end{frame}

%% Beginn Hauptteil
\section{Einführung}
\subsection{Was ist \LaTeX{}}
\begin{frame}{Was ist \LaTeX{}}
\begin{itemize}
	\item \TeX{} = Schriftsatzsystem vom Informatik Professor Donald Knuth
	\item \LaTeX{} = Weiterentwicklung durch Leslie Lamport ($\rightarrow$ \textbf{La}mport \textbf{TeX})
	\item vereinfacht TeX u.a. durch die Benutzung von Makros
	\item Entwicklung dauert seit den 1990ern an, allerdings werden nur noch kleine Änderungen vorgenommen, weil \LaTeX{} schon \textit{relativ fertig} ist
\end{itemize}
\end{frame}


\subsection{Vorteile}
\begin{frame}
\textbf{Vorteile:}
\begin{itemize}
	\item open-source \& kostenlos
	\item professionelles Aussehen
	\item Automatisierung von Nummerierungen oder Verweisen
	\item Mathematischer Formelsatz
	\item Fokus auf Inhalt
	\item sehr performant, egal wie groß das Dokument ist
\end{itemize}
\end{frame}


\subsection{Schwächen}
\begin{frame}
\textbf{Schwächen:}
\begin{itemize}
	\item ... die gewohnten Arbeitsabläufe funktionieren hier nicht mehr wie bei Word, weil \LaTeX{} kein WYSIWYG-Programm ist!
	\item Anfangs ist ein mühsames Umdenken nötig, aber je mehr man damit arbeitet, desto einfacher wird es
	\item \LaTeX{} lohnt sich \textit{(aus meiner Erfahrung)} nur für längere Dokumente, wie wissenschaftliche Arbeiten, Bücher, ...
	\item Für kurze Dokumente kann Word genau so gut aussehen und vor allem zeiteffizienter sein.
\end{itemize}
\end{frame}


\subsection{Installation}
\begin{frame}{Installation: Distribution}
Zunächst benötigt man eine \LaTeX{}-Distribution
\begin{itemize}
	\item = \LaTeX{}-Kernel und Sammlung an Paketen
	\item \textbf{MiKTeX} (Empfehlung), TexLive, ...
	\item MikTeX installiert u.a. fehlende Pakete automatisch, ist relativ einfach gehalten und betriebssystemunabhängig
\end{itemize}
\end{frame}


\begin{frame}{Installation: Editor}
Da die Distribution wie bei Linux als Konsole im Hintergrund arbeitet, benötigt man noch eine Schnittstelle zum Benutzer, die eine einfachere Bedienung ermöglicht.
\begin{itemize}
	\item = Bearbeitungsprogramm
	\item \textbf{Texmaker} (Empfehlung), Texworks, ...
	\item Texmaker hat einen Darkmode und viele weitere nützliche Funktionen
\end{itemize}
Zur Installation von Texmaker und MiKTeX kann dieses \href{https://www.youtube.com/watch?v=aTOfbfJvUig}{Videotutorial} befolgt werden. \par\medskip
Aktuelle Versionen: \href{https://miktex.org/download}{MiKTeX}, \href{https://www.xm1math.net/texmaker/download.html}{Texmaker}
\end{frame}


\begin{frame}{Alternative: Online-Editor}
Der kostenlose Online-Dienst \href{https://de.overleaf.com/}{Overleaf} bietet eine Cloud-basierte Bearbeitung von \LaTeX{}-Dokumenten mit (mehr oder weniger) Echtzeitverarbeitung. \par\medskip
Hierzu ist nur die Erstellung eines Kontos und keine Installation notwendig. \par\medskip
Für das spätere Arbeiten eignen sich beide Varianten.
\end{frame}


\section{Basics}
\subsection{Grundgerüst}
\begin{frame}[fragile]{Grundgerüst}
Jedes \LaTeX{}-Dokument muss mindestens aus folgenden Befehlen bestehen:
\begin{lstlisting}
\documentclass{article}
	% Präambel: Konfiguration des Dokuments
\begin{document}
	Hallo Welt! % Dokumentinhalt
\end{document}
\end{lstlisting}
Befehle sind dabei immer so aufgebaut:
\begin{lstlisting}
\befehl[option1, opt2, ...]{argument1}{arg2}{...}
\end{lstlisting}
\end{frame}


\subsection{Leerzeichen}
\begin{frame}[fragile]{Leerzeichen}
%\beamheading{Leerzeichen}
Bei \LaTeX{} ist es völlig egal, wie viele Leerzeichen gesetzt werden. Alles wird wie ein Leerzeichen angesehen. 
\begin{lstlisting}
Bei \LaTeX{} ist es völlig egal,    wie viele
   Leerzeichen   gesetzt    werden.
\end{lstlisting}
Für zusätzliche Abstände gibt es aber Befehle.
\end{frame}


\subsection{Reservierte Zeichen}
\begin{frame}[fragile]{Reservierte Zeichen}
Folgende Zeichen sind von \LaTeX{} reserviert: \texttt{\# \$ \% \^{} \& \_ \{ \} \~{} \textbackslash{}} \par\medskip
Wenn sie im Text benutzt werden wollen, muss folgende Schreibweise beachtet werden: \par\medskip
{\footnotesize
\begin{tabular}{|c|l|l|}
\hline
Zeichen & Bedeutung & Benutzung im Text \\
\hline
\texttt{\textbackslash} & Start Befehl & \verb|\textbackslash| \\
\texttt{\textbackslash\textbackslash} & Neue Zeile (=\verb|\newline|) & \\
\texttt{\$} & Mathe-Modus & \verb|\$| \\
\texttt{\&} & Tabulator & \verb|\&| \\
\texttt{\%} & Kommentar & \verb|\%| \\
\texttt{\#} & Raute & \verb|\#| \\
\texttt{\~{}} & Tilde & \verb|\~{}| \\
\texttt{\_} & Tiefstellung & \verb|\_| \\
\texttt{\^{}} & Hochstellung & \verb|\^{}| \\
\texttt{\{ \}} & Argumente & \verb|\{ \}| \\
\texttt{[ ]} & Optionen & \verb|$[ ]$| \\ \hline
\end{tabular} 
}
\end{frame}


\begin{frame}[fragile]{Bindestriche}
Ein horizontaler Bindestrich kann vielfältig angewendet werden, da seine Länge variabel ist: \par\medskip
{\footnotesize
\begin{tabular}{|l|l|l|}
\hline
Binde-strich & \verb|Binde-strich| & Für Wörter \\
6--16 Uhr & \verb|6--16 Uhr| & Für Uhrzeiten \\
--10 & \verb|--10| & Für Zahlen, aber \verb|$-10$| ist besser \\
ja -- oder nein & \verb|ja -- oder nein| & \\
ja --- oder nein & \verb|ja --- oder nein| & \\ \hline
\end{tabular}
}
\end{frame}


\subsection{interne Parameter \& Befehle}
\begin{frame}[fragile]{Längenparameter}
\LaTeX{} hat einige interne Parameter, die das Aussehen des Dokuments bestimmen. Hier sind ein paar wichtige: \par\medskip
\begin{lstlisting}
\parindent 		% Einzug bei neuem Absatz
\parskip 		% Abstand nach Absatz
\baselineskip 	% Abstand zwischen Zeilen
\textwidth 		% Breite des Textes auf einer Seite
\linewidth 		% Länge einer Zeile in aktueller Umgebung
% ...
\end{lstlisting}
Die Parameter haben angepasste Standardwerte und müssen somit nicht verändert werden. Falls aber doch geht das so: \par\medskip
\begin{lstlisting}
\setlength{\parindent}{0pt} 	% Setzt Länge zu 0pt
\addtolength{\parindent}{10mm} % Addiert 10mm zu Länge
\end{lstlisting}
\end{frame}


\begin{frame}[fragile]{Längeneinheiten}
Man kann je nach Situation unterschiedliche Einheiten verwenden: \par\medskip
\begin{tabular}{ll}
\texttt{in} & inches bzw. Zoll \\
\texttt{mm} & Millimeter \\
\texttt{cm} & Centimeter \\
\texttt{pt} & points (ca. $1/72$ inch) \\
\texttt{em} & Breite eines ''M'' in aktueller Schriftart \\
\texttt{ex} & Höhe eines ''x'' in aktueller Schriftart
\end{tabular}
\end{frame}


\begin{frame}[fragile]{Horizontale Abstände}
Test
\end{frame}


\begin{frame}[fragile]{Vertikale Abstände}
Test
\end{frame}


\subsection{Umgebungen}
\begin{frame}[fragile]{Umgebungen}
%\beamheading{Umgebungen}
Eine Umgebung (\texttt{environment}) muss immer geöffnet \textbf{und} geschlossen werden. \par\medskip
In die Umgebung kommt dann der Inhalt, der von der Umgebung beeinflusst wird. \par\medskip
\begin{lstlisting}
\begin{umgebung}[optionen]
	% dieser Text/Code wird von der Umgebung beeinflusst
\end{umgebung}
\end{lstlisting}
\end{frame}


\subsection{Kommentare}
\begin{frame}[fragile]{Kommentare}
%\beamheading{Kommentare}
Kommentare können mit dem \texttt{\%}-Zeichen eingefügt werden. \\
Alles was danach kommt, wird von \LaTeX{} nicht beachtet. \par\medskip
Es gibt \textbf{keine} Blockkommentare. \par\medskip
\begin{lstlisting}
% Dies ist ein Kommentar
%%%%% Die Anzahl ist egal
kein Kommentar	% Kommentar % immernoch Kommentar
\end{lstlisting}
\end{frame}


\subsection{Kompilieren}
\begin{frame}[fragile]
\beamheading{Dokumente Kompilieren}
Nachdem man sein erstes Test-Dokument geschrieben hat, kann man das Dokument einfach über den Knopf "\textit{PDFLaTeX}"{} kompilieren.
\begin{lstlisting}
\documentclass{article}
\begin{document}
	Hallo Welt! 
\end{document}
\end{lstlisting}
\end{frame}


\section{Textgestaltung}
\subsection{Textgrößen}
\begin{frame}[fragile]{Textgrößen}
Durch verschiedene Größen, können z.B. Überschriften vom restlichen Text getrennt werden. Die Größen werden fast immer automatisch durch \LaTeX{} festgelegt. \par\medskip
\begin{tabular}{ll}
\verb|\tiny| & {\tiny Mikroschrift} \\
\verb|\scriptsize| & {\scriptsize Tiefstellung} \\
\verb|\footnotesize| & {\footnotesize Fußnoten} \\
\verb|\small| & {\small klein} \\
\verb|\normalsize| & {\normalsize normal} \\
\verb|\large| & {\large groß} \\
\verb|\Large| & {\Large größer} \\
\verb|\LARGE| & {\LARGE sehr groß} \\
\verb|\huge| & {\huge riesig} \\
\verb|\Huge| & {\Huge sehr riesig}
\end{tabular}
\end{frame}


\subsection{Texthervorhebung}
\begin{frame}[fragile]{Texthervorhebung}
\begin{tabular}{ll}
\verb|\textit{}| & \textit{kursiv} (it=italic)\\
\verb|\textsl{}| & \textsl{angewinkelt} \\
\verb|\emph{}| & \emph{betont} \quad (da sie nicht wirklich unterscheidbar sind, \\ & würde ich nur \verb|\textit{}| verwenden) \\
\verb|\textbf{}| & \textbf{fett} \\
\verb|\textsc{}| & \textsc{kapitälchen} (sc=small caps) \\
\verb|\textrm{}| & \textrm{serifenschrift} (rm=roman) \\
\verb|\textsf{}| & \textsf{serifenlos} (sf=serif) \\
\verb|\texttt{}| & \texttt{schreibmaschine} 
\end{tabular}
\end{frame}


\subsection{Textausrichtung}
\begin{frame}[fragile]{Textausrichtung}
Test
\end{frame}


\section{Strukturierung von \LaTeX{}-Dokumenten}
\subsection{Präambel}
\begin{frame}[fragile]{Präambel}
\begin{itemize}
	\item Präambel entspricht einer Header-Datei bei C
	\item befindet sich zwischen \lstinline|\documentclass[]{}| und \lstinline|\begin{document}|
	\item beinhaltet alle Pakete, die benötigt werden und
	\item konfiguriert das gesamte Dokument
	\item für die T1000/2000/... existiert eine gute Vorlage
\end{itemize}
\end{frame}


\subsection{Abschnitte}
\begin{frame}[fragile]{Abschnitte}
Abschnitte strukturieren das Dokument und werden automatisch ins Inhaltsverzeichnis eingefügt. Es gibt verschiedene Tiefen von Abschnitten in folgender Reihenfolge (höchste zu kleinste):
\begin{lstlisting}
%\chapter{}
%\section[Titel im Inhaltsverzeichnis]{Titel}
%\subsection{Titel}
%\subsubsection{Titel}
%
%\abschnitt*{} => erscheint nicht im Inhaltsverzeichnis
\end{lstlisting}
Es gibt noch weitere Abschnitte, aber diese werden sehr selten benutzt und deshalb hier nicht behandelt.
\end{frame}


\subsection{Verzeichnisse}
\begin{frame}[fragile]{Verzeichnisse}
In \LaTeX{} werden alle Verzeichnisse automatisch beim Kompilieren erstellt.
\begin{lstlisting}
\tableofcontents 	% Hier wird das Inhaltsverzeichnis erstellt
\listoffigures 		% Hier wird das Abbildungsverz. erstellt
\listoftables 		% Hier wird das Tabellenverz. erstellt
\end{lstlisting}
\end{frame}


\subsection{Grafiken}
\begin{frame}[fragile]{Einfügen von Grafiken}
\LaTeX{} unterstützt von Haus aus nur \texttt{.eps}-Dateien. \par\medskip
Mit dem Paket \lstinline|\usepackage{graphicx}| können aber auch \texttt{.jpg}-, \texttt{.png}- und \texttt{.pdf}-Dateien eingebunden werden.
\begin{lstlisting}
\includegraphics[width=\textwidth]{images\Bild.png}
% oder
\includegraphics[scale=0.5]{images\Bild.png}
\end{lstlisting}
Bei der Suche nach Bildern für eine Arbeit sollten möglichst hochauflösende \texttt{jpg}- oder wenn möglich sogar \texttt{pdf}-Dateien verwendet werden, sonst leidet das Aussehen.
\end{frame}


\subsection{Abbildungen}
\begin{frame}[fragile]{Abbildungen}
Damit noch eine Bildunterschrift eingefügt und später im Text referenziert werden kann, muss die \texttt{figure}-Umgebung benutzt werden: 
\begin{lstlisting}
\begin{figure}[platzierung]
	\centering	% Zentriert die Abbildung
		\includegraphics{images\Bild.png}
	\caption{Bildunterschrift}
	\label{Name} % Referenzmarke
\end{figure}
\end{lstlisting}

Für die \texttt{platzierung} gibt es folgende Angaben: (\texttt{float}-Paket nötig!) \par\medskip
{\footnotesize
\begin{tabular}{|l|l|}
\hline 
\texttt{h} & Platzierung ungefähr an der Stelle im Code (\textit{here}) \\ 
\hline 
\texttt{t} & Platzierung oben auf der Seite (\textit{top})\\
\hline 
\texttt{b} & Platzierung auf extra Seite \\ 
\hline 
\texttt{!} & überschreibt alle internen Parameter \\ 
\hline 
\texttt{H} & exakte Platzierung an der Stelle im Code (\textit{HERE!}) \\ 
\hline
\end{tabular}
}
\end{frame}


\subsection{Referenzierung}
\begin{frame}[fragile]{Referenzierung}
Referenzmarken können mit \lstinline|\label{markenname}| erstellt werden. Wo der Marker im Code steht ist egal, er muss nur in der Nähe von dem Inhalt stehen, der referenziert werden soll. \par\medskip
Um später die Marke wiederzufinden, sollte der Name gut gewählt werden.
Folgende Nomenklatur wird häufig verwendet: \\
\begin{lstlisting}
\label{cha:name}	% Für Chapters
\label{sec:name}	% Für Sections
\label{sub:name}	% Für Subsections
\label{fig:name}	% Für Abbildungen
\label{eq:name}		% Für Gleichungen ...
\end{lstlisting}
Die spätere Referenzierung im Text sieht dann z.B. so aus:
\lstinline|\ref{cha:name}|
\end{frame}


\subsection{Tabellen}
\begin{frame}[fragile]{Tabellen}
Diese können mit der \texttt{tabular}-Umgebung erzeugt werden. Texmaker bietet allerdings auch einen Tabellen-Assistenten an, der Tabellen erzeugen kann. \\
\begin{lstlisting}
\begin{tabular}{spalten}
\hline % horizontale Linie
Spalte1/Zeile1 & Spalte2/Zeile1 & ... \\
\hline
Spalte1/Zeile2 & Spalte2/Zeile2 & ... \\
...
\end{tabular}
\end{lstlisting}
Die Spaltenangabe besteht aus folgenden Angaben: \par\medskip
{\footnotesize
\begin{tabular}{|c|l|}
\hline
\texttt{l} & linksbündige Spalte \\
\hline
\texttt{c} & zentrierte Spalte \\
\hline
\texttt{p\{Breite\}} & für Textabsätze \\
\hline
\texttt{|} & senkrechte Linie \\
\hline
\texttt{||} & doppelte Linie \\
\hline
\end{tabular}
}
\end{frame}


\begin{frame}[fragile]{Tabellen}
Folgende Befehle können in der \texttt{tabular}-Umgebung angewendet werden: \par\medskip
{\footnotesize
\begin{tabular}{|c|l|}
\hline
\texttt{\&} & trennt Spalten \\ \hline
\texttt{\textbackslash\textbackslash} & neue Zeile \\ \hline
\verb|\hline| & horizontale Linie\\ \hline
\verb+\newline+ & neue Zeile in p-Spalte \\ \hline
\end{tabular}
}

\beamheading{Beispiele}
\begin{columns}
\begin{column}{0.48\textwidth} % Beispiel 1
\begin{lstlisting}
\begin{tabular}{rlc}
	1 & 2 & 3 \\ \hline
	4 & 5 & 6 \\
	7 & 8 & 9 \\
\end{tabular}
\end{lstlisting} \end{column}

\begin{column}{0.48\textwidth} % Beispiel 2
\begin{lstlisting}
\begin{tabular}{l|c|r||}
	\hline
	1 & 2 & 3 \\ \hline
	4 & 5 & 6 \\ \hline
	7 & 8 & 9 \\ \hline \hline
\end{tabular}
\end{lstlisting} \end{column}
\end{columns} 

\begin{columns}
\begin{column}{0.48\textwidth} % Beispiel 1
{\footnotesize
\begin{tabular}{rlc}
	1 & 2 & 3 \\
	4 & 5 & 6 \\
	7 & 8 & 9 \\
\end{tabular} } \end{column}

\begin{column}{0.48\textwidth} % Beispiel 2
{\footnotesize
\begin{tabular}{l|c|r||}
	\hline
	1 & 2 & 3 \\ \hline
	4 & 5 & 6 \\ \hline
	7 & 8 & 9 \\ \hline \hline
\end{tabular} } \end{column}
\end{columns}
\end{frame}


\begin{frame}[fragile]{Tabellen}
Ähnlich wie mit der \texttt{figure}-Umgebung kann auch eine Tabelle mit der \texttt{table}-Umgebung dargestellt werden:
\begin{columns}
	\begin{column}{0.48\textwidth}
\begin{lstlisting}
\begin{table}
	\centering
		\begin{tabular}{c|c|c}
		1 & 2 & 3 \\ \hline
		4 & 5 & 6 \\ \hline
		7 & 8 & 9 \\ \hline
		\end{tabular}
	\caption{Eine Tabelle}
\end{table}	
\end{lstlisting}
	\end{column}
	\begin{column}{0.48\textwidth}
\begin{table}
	\centering
		\begin{tabular}{c|c|c}
		1 & 2 & 3 \\ \hline
		4 & 5 & 6 \\ \hline
		7 & 8 & 9 \\ 
		\end{tabular}
	\caption{Eine Tabelle}
\end{table}	
	\end{column}
\end{columns}
Tabellen und Abbildungen haben dabei ihre eigenen Zähler.
\end{frame}


\subsection{Listen}
\begin{frame}[fragile]{Listen}
Es gibt drei verschiedene Listenumgebungen:
	\begin{itemize}
		\item \texttt{itemize}
		\item \texttt{enumerate}
		\item \texttt{description}
	\end{itemize}
\begin{columns}
	\begin{column}{0.31\textwidth} % itemize
\begin{lstlisting}[basicstyle=\tiny\ttfamily]
\begin{itemize}
	\item erster Punkt
	\begin{itemize}
		\item Unterpunkt
	\end{itemize}
	\item zweiter Punkt
\end{itemize}
\end{lstlisting}
	\end{column}
	\begin{column}{0.31\textwidth} % enumerate
\begin{lstlisting}[basicstyle=\tiny\ttfamily]
\begin{enumerate}
	\item erster Punkt
	\begin{enumerate}
		\item Unterpunkt
	\end{enumerate}
	\item zweiter Punkt
\end{enumerate}	
\end{lstlisting}
	\end{column}
	\begin{column}{0.31\textwidth} % description
\begin{lstlisting}[basicstyle=\tiny\ttfamily]
\begin{description}
	\item[Erstens:] ...
	\item[Zweitens:] ...
\end{description}	
\end{lstlisting}
	\end{column}
\end{columns}
	
\begin{columns}
	\begin{column}{0.31\textwidth} % itemize
		\begin{itemize}
			\item erster Punkt
			\begin{itemize}
				\item Unterpunkt
			\end{itemize}
			\item zweiter Punkt
		\end{itemize}
	\end{column}
	\begin{column}{0.31\textwidth} % enumerate
		\begin{enumerate}
			\item erster Punkt
			\begin{enumerate}
				\item Unterpunkt
			\end{enumerate}
			\item zweiter Punkt
		\end{enumerate}	
	\end{column}
	\begin{column}{0.31\textwidth} % description
		\begin{description}
			\item[Erstens:] ...
			\item[Zweitens:] ...
		\end{description}	
	\end{column}
\end{columns}
\end{frame}


\section{Mathematische Ausdrücke}
\begin{frame}[fragile]{}
Um diese zu benutzen ist das \texttt{amsmath}-Paket nützlich. \par\medskip
\LaTeX{} unterscheidet zwei Arten der Formeleingabe:
\begin{itemize}
	\item Inline: innheralb einer Zeile im Text
	\item Display: als eigene Gleichung vom Text getrennt
\end{itemize} \par\bigskip

\begin{columns}
	\begin{column}{0.48\textwidth} % Inline
	Inline: \\
\begin{lstlisting}
Hier $x^2 - 1 = 0$ im Text
\end{lstlisting}
	\end{column}
	\begin{column}{0.48\textwidth} % Display
	Display: \\
\begin{lstlisting}
$$x^2 - 1 = 0$$
\end{lstlisting}
	\end{column}
\end{columns}

\begin{columns}
	\begin{column}{0.48\textwidth} % Inline
Hier $x^2 - 1 = 0$ im Text
	\end{column}
	\begin{column}{0.48\textwidth} % Display
$$x^2 - 1 = 0$$
	\end{column}
\end{columns}
\end{frame}


\subsection{Gleichungsumgebungen}
\begin{frame}[fragile]{\texttt{equation}-Umgebung}
Mittels der \texttt{equation}-Umgebung ist \texttt{displaymath} auch wie in der \texttt{figure}-Umgebung möglich:

\begin{columns}
	\begin{column}{0.48\textwidth} % Gleichung
\begin{lstlisting}
\begin{equation}
a^2 + b^2 = c^2
	\label{eq:pythagoras}
\end{equation}
siehe Gleichung
\eqref{eq:pythagoras}
\end{lstlisting}
	\end{column}
	\begin{column}{0.48\textwidth} % Gleichung
\begin{equation}
a^2 + b^2 = c^2
	\label{eq:pythagoras}
\end{equation}
siehe Gleichung
\eqref{eq:pythagoras}
	\end{column}
\end{columns}

Ist die Nummerierung nicht gewünscht, kann \\
\lstinline|\begin{equation*} ... \end{equation*}| \\
verwendet werden.
\end{frame}


\begin{frame}[fragile]{\texttt{align}-Umgebung}
Die \texttt{align}-Umgebung sorgt dafür, dass mehrere Gleichungen übereinander an einem Zeichen angeordnet werden können. Dazu wird das \texttt{\&}-Zeichen verwendet.
\begin{columns}
	\begin{column}{0.48\textwidth}
\begin{lstlisting}
\begin{align}
a^2 + b^2 &= c^2 \\
a^2 &= c^2 - b^2
\end{align}
\end{lstlisting}		
	\end{column}
	\begin{column}{0.48\textwidth}
\begin{align}
a^2 + b^2 &= c^2 \\
a^2 &= c^2 - b^2
\end{align}	
	\end{column}
\end{columns}
Bei nicht gewünschter Nummerierung gilt wieder: \\
\lstinline|\begin{align*} ... \end{align*}|
\end{frame}


\subsection{Symbole}
\label{sub:symb}
\begin{frame}[fragile]{Symbole}
Es gibt \textit{sehr} viele Symbole in \LaTeX{}, was es leicht macht den Überblick zu verlieren. Es ist daher nützlich ein \textit{Cheat-Sheet} zu benutzen oder die gewünschten Befehle einfach zu googeln. \par\medskip
Im \href{https://en.wikibooks.org/wiki/LaTeX/Mathematics}{Wikibooks} (oder \href{https://en.wikipedia.org/wiki/List_of_mathematical_symbols_by_subject}{Wikipedia}) ist eine Liste an Symbolen zu finden. Über \href{http://detexify.kirelabs.org/classify.html}{Detexify} kann ein bestimmtes Symbol auch über Zeichnen gefunden werden.\par\medskip
Folgende Symbole können direkt eingegeben werden: \par\medskip
\verb#+ - = ! / ( ) [ ] < > | ' :#
\end{frame}


\subsection{Hoch- \& Tiefstellung}
\begin{frame}[fragile]{Hoch- \& Tiefstellung}
Ausdrücke können mit \verb|^| hoch- und mit \verb|_| tiefgestellt werden. \\
Wenn Ausdrücke aus mehr als einem Zeichen bestehen muss dieser mit \verb|{...}| eingeklammert werden. \par\medskip
\begin{columns}
	\begin{column}{0.48\textwidth}
\begin{lstlisting}
a_{n+1} = a_n^2 + (n-1)^2
\end{lstlisting} 
	\end{column}
	\begin{column}{0.48\textwidth}
	\centering
$a_{n+1} = a_n^2 + (n-1)^2$
	\end{column}
\end{columns}
\end{frame}


\subsection{Brüche}
\begin{frame}[fragile]{Brüche}
Brüche können sowohl mit \lstinline|\frac{zähler}{nenner}| (entscheidet abhängig von Umgebung, wie Bruch dargestellt werden soll) und mit \lstinline|\dfrac{zähler}{nenner}| (Bruch im \texttt{displaystyle}). \par\medskip
Wenn kein Platz vorhanden ist, kann ein Bruch auch mit \texttt{zähler/nenner} erzeugt werden.
\begin{columns}
	\begin{column}{0.64\textwidth}
\begin{lstlisting}
\frac{n!}{k!(n-k)!} = \binom{n}{k}
\end{lstlisting}
\begin{lstlisting}
\frac{\frac{1}{x}+\frac{1}{y}}{y-z}
\end{lstlisting}
\begin{lstlisting}
1+\dfrac{1}{1+\dfrac{1}{1+\dfrac{1}{1+\dots} } }
\end{lstlisting}
	\end{column}
	\begin{column}{0.34\textwidth}
	\centering
$$\frac{n!}{k!(n-k)!} = \binom{n}{k}$$
$$\frac{\frac{1}{x}+\frac{1}{y}}{y-z}$$
$$1+\frac{1}{1+\frac{1}{1+\frac{1}{1+\dots} } }$$
	\end{column}
\end{columns}
\end{frame}


\subsection{Summen \& Integrale}
\begin{frame}[fragile]{Summen \& Integrale}
Hier unterscheiden sich wieder die zwei Arten der Formeleingabe: \par\medskip
\begin{columns}
	\begin{column}{0.74\textwidth}
\begin{lstlisting}
% Inline Summe im Text
\sum_{n=1}^{\infty}\frac{1}{n^2}
\end{lstlisting}
\begin{lstlisting}
% Display-Stil in Umgebung
\sum_{n=1}^{\infty}\frac{1}{n^2}
\end{lstlisting}
\begin{lstlisting}
% Inline Integral im Text
\int_{-\infty}^{\infty} x^2 \mathrm{d}x
\end{lstlisting}
\begin{lstlisting}
% Display-Stil in Umbebung
\int_{-\infty}^{\infty} x^2 \mathrm{d}x
\end{lstlisting}
	\end{column}
	\begin{column}{0.24\textwidth}
	\centering
$\sum_{n=1}^{\infty}\frac{1}{n^2}$ \vfil
$$\sum_{n=1}^{\infty}\frac{1}{n^2}$$ \vfil
$\int_{-\infty}^{\infty} x^2 \mathrm{d}x$ \vfil
$$\int_{-\infty}^{\infty} x^2 \mathrm{d}x$$ \vfil
	\end{column}
\end{columns}
Auf weitere Befehle kann nicht weiter eingegangen werden. Die meisten haben aber eine einfache Syntax und können leicht über die erwähnten Nachschlagewerke gefunden werden.
\end{frame}


\subsection{Klammern}
\begin{frame}[fragile]{Korrekte Klammersetzung}
Die Größe der Klammern wird von \LaTeX{} automatisch angepasst, allerdings muss folgendes beachtet werden: \par\bigskip
	\begin{minipage}{0.74\textwidth}
\begin{lstlisting}
( \frac{x^2}{y^3} ) % Falsch!
\end{lstlisting} \bigskip
\begin{lstlisting}
\left( \frac{x^2}{y^3} \right) % Korrekt
\end{lstlisting} \bigskip
\begin{lstlisting}
\left\{ \frac{x^2}{y^3} \right\} % Korrekt
\end{lstlisting} \bigskip
\begin{lstlisting}
\left. \frac{x^2}{2} \right|_0^1 % Korrekt
\end{lstlisting} \bigskip
	\end{minipage}
	\begin{minipage}{0.24\textwidth}
$$( \frac{x^2}{y^3} )$$ 
$$\left( \frac{x^2}{y^3} \right)$$ 
$$\left\{ \frac{x^2}{y^3} \right\}$$ 
$$\left. \frac{x^2}{2} \right|_0^1$$ 
	\end{minipage}
\end{frame}


\section{Bibliographie \& Zitieren}
\begin{frame}[fragile]{Bibliographie}
\LaTeX{} kann Literaturverzeichnisse mit einem Tool namens \textbf{BibTeX} organisieren und erstellen. \par\medskip
Die Quelleneinträge werden über eine Art "Datenbank" verwaltet und können anschließend im Dokument referenziert werden. \par\medskip
Die Literatureinträge werden in der Präambel eingebunden: \par\medskip
\begin{lstlisting}
\usepackage[backend=biber]{biblatex}
\addbibresource{quellen.bib}
\end{lstlisting}
Das Literaturverzeichnis wird dann durch den Befehl \lstinline|\printbibliography| an der Stelle im Code erzeugt.
\end{frame}


\begin{frame}[fragile]{Zitieren}
Ein Literatureintrag hat immer einen \textit{Typ} (z.B.: Buch), ein \textit{Kürzel} für den schnellen internen Aufruf in \LaTeX{} und diverse \textit{Angaben} zur Quelle.
\begin{lstlisting}
% Beispielinhalt einer .bib-Datei
@book{kürzel, % Für den Aufruf in LaTeX
	title={LaTeX-Tutorium},
	author={P., Tim},
	year={2022}
	% mögliche weitere Angaben
}
\end{lstlisting}
Soll die Quelle jetzt referenziert werden, verwendet man einfach den Befehl \lstinline|\cite{kürzel}| an der Stelle im Text. \par\medskip
Die Einträge können über Literatur-Tools, Online oder auch manuell erstellt werden.
\end{frame}


\section{Besonderheiten}
\subsection{Einheiten}
\begin{frame}[fragile]{SI-Einheiten}
Mithilfe des \texttt{SI}-Pakets können Einheiten konsistent eingegeben werden. \par\medskip

\end{frame}


\subsection{Abkürzungen}
\begin{frame}[fragile]{Abkürzungen}
Test
\end{frame}


\subsection{Fußnoten}
\begin{frame}[fragile]{Fußnoten}
Test
\end{frame}


\section{Hilfsmittel}
\begin{frame}[fragile]{Wo finde ich Hilfe?}
\begin{itemize}
	\item Die Kurzeinführung \href{http://mirrors.ibiblio.org/CTAN/info/german/LaTeX2e-Kurzbeschreibung/l2kurz.pdf}{"Docu-l2kurz-german"} ist ca. 60 Seiten groß und umfasst alle Basics
	\item Google ist dein Freund: Wenn Probleme auftreten, einfach die Fehlermeldung in eine Suchmaschine schmeißen...
	\item Zusammenstellung der meisten Befehle, mit deutscher Erklärung: \href{http://www.weinelt.de/latex/index.html}{CTAN}
	\item Diverse Cheat-Sheets erleichtern das Arbeiten und liegen in meinem  \href{https://ctan.org/}{Git-Repo}
	\item Falls nichts funktioniert, könnt ihr mich unter \verb|pruess.tim-it20@it.dhbw-ravensburg.de| erreichen
\end{itemize}
\end{frame}


\section{Pakete mit Kurzbeschreibung}
\begin{frame}[fragile]{Pakete mit Kurzbeschreibung}
\begin{description}
	\item[paket] test
	%\item[\verb|\usepackage[onehalfspacing]{setspace}|] Verändert den Standardmäßigen Zeilenabstand auf 1,5-fach
	%\item[\verb|\usepackage{amsmath, amsthm}|] Erweiterung für den Mathematik-Satz
\end{description}
\end{frame}


\section*{Literaturverzeichnis}
\begin{frame}{Literatur}
\nocite{*}					% Alle Literatureinträge einblenden
\printbibliography			% Erstellen des Literaturverzeichnisses
\end{frame}

\end{document}